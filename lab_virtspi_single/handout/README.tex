\documentclass{article}

\usepackage{../../common/labconfig}
\usepackage{../../common/labstyle}

\lhead{Virtual SPI Lab}

\begin{document}

\section{Introduction}

In this lab, you will write a software-based "bit banged" SPI implementation,
which will read and write data to a simulated peripheral.

\begin{figure}[H]

	\centering

	\begin{tabular}{r|l}

		Part & Due Date \\ \hline\hline
		Code & Apr. 03\\ \hline

	\end{tabular}

	\caption{Table of due dates for each part.}

\end{figure}

\tableofcontents

\section{How to Submit }

When your code is ready to turn in, please submit only your \texttt{main.cpp}
file via Moodle. Note that you should write all code that you need within
\texttt{main.cpp}. As before, your code must compile to receive any credit.

\section{Simulated Sensor}

The peripheral devices you will be interfacing with during this lab is a
simulated "sensor". We leave it up to your imagination as to what it senses.
This device works very similarly to the LIS3DH, but is greatly simplified. It
has only two registers, which are both read-only.

The first register is \texttt{WHO\_AM\_I}, which is at address \texttt{0x0F}.
When read, it will always return \texttt{0x34}.

The second register is \texttt{DATA}, which is at address \texttt{0x10}. This
register will return the sensor "reading". Note that the sensor readings are
actually hard-coded, however keep in mind that they will be modified to
different hard-coded values during grading.

Much like the LIS3DH, a register is read by transmitting it's 8-bit address via
MOSI. The 8-bit register contents will then be returned via MISO.

\subsection{Program Requirements}

Your program should first query the \texttt{WHO\_AM\_I} register and verify it
returns the correct value. If not, then your program should display an error
message (on standard out) and exit.

If the \texttt{WHO\_AM\_I} register can be verified, then your program should
read the \texttt{DATA} register \textbf{four} times, printing the raw value
received each time to standard out, then exit.

\subsection{Grading}

Your code will be inspected for style and correctness by a human reader. This
aspect of grading will be fairly lenient and mostly for the purpose of giving
you useful feedback. However you may still lose points for egregious stylistic
problems, or failing to write code that clearly attempts to solve the problem
at hand.

Your code will also be run against the same simulated sensor as you are given
in the project skeleton, however the hard-coded sensor "readings" will be
changed to different values.

Additionally, your code will be run against a version of the simulated sensor
which is defective, and reports an incorrect value when the \texttt{WHO\_AM\_I}
register is read.

The correctness of your code will be judged by considering both your printouts,
as well as the values sigrok detects on the SPI bus. Note that the error value
\texttt{0xEE} or null \texttt{0x00} can appear on the bus as the "dummy value"
which one side or the other "sends" while receiving data. This is not a cause
for concern.

If your code does not compile and run on the CSCE linux lab computers, you will
not receive credit.

\section{Rubric}

\begin{itemize}

	\item 10 points - Code style

	\item 10 points - Correct data is transmitted

	\item 10 points - Correct sensor readings are received and printed

	\item 10 points - Correct checking of \texttt{WHO\_AM\_I} register

\end{itemize}

\textbf{Maximum number of points possible: 40.}

Keep in mind that some items not listed on the rubric may cause you to loose
points, including cheating, submitting code which is inconsistent with what you
have demonstrating, plagiarizing code or reflection content, etc.


\end{document}
